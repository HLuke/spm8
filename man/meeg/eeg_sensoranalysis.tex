\chapter{Analysis in sensor space \label{Chap:eeg:sensoranalysis}}

Second-level analyses for M/EEG are used to make inferences about the population from which subjects are drawn. This chapter describes how to perform second-level analyses of EEG/MEG or time-frequency (TF) data. This simply requires transforming data from \texttt{filename.mat} and \texttt{filename.dat}
format to image files (NIfTI format). Once the data are in image format second-level analyses for M/EEG are procedurally identical to 2nd level analyses for fMRI.
We therefor refer the reader to the fMRI section for further details of this last step.
\\
\\
To display the MEG/EEG/TF data the images are normalised to a standard size. Space (X,Y) dimensions are stretched to be displayed gracefully in SPM glass brain (note that it does not correspond to real \texttt{mm} coordinates). Time dimension is renormalised to be expressed as percent, from 0 to 100.

This means that the units are in a normalised space. Hopefully we aim at improving the visualisation in a near future.
\\
\\
In the GUI ``other'' pull down menu, select the function \texttt{Convert to images}. This will ask you to select the \texttt{filename.mat} of the data you would like to convert to images.
\\
\\
If you supply a result of a time-frequency decomposition you will first be asked to reduce your data from a 4D data (space(X,Y), time, frequency) to either a 3D image (space(X,Y), time) or a 2D time-frequency image (time, frequency). The prompt asks you over which dimension you wish to average your data.
\\
\\
If you select ``electrodes'' you will be asked to enter a vector of channels you wish to average over. Next you will be prompted for a ``region number''. This allows you to average over different electrodes/sensors to create different analyses for more than one region. The image will be created in a new directory with the name \texttt{<region number>ROI\_TF\_trialtype\_<condition number>}. The image that will be created will be a 2D image where the dimensions are time and frequency. Before proceeding to the second-level analysis these images should be smoothed. This is performed using the \textsc{smooth} function in the drop-down ``Other'' menu. Once the images have been smoothed one can proceed to second level analyses.
\\
\\
If you select ``frequencies'' you will be asked to specify the frequency range over which you wish to average. You will then be prompted to enter the output dimensions of the (interpolated scalp) image that will be produced. Typically we suggest using 64.  You will then be asked whether you want to interpolate or remove bad channels from your images. If you chose interpolate then the image will interpolate missing channels. This is the preferred option. You will then be asked for the pixel dimensions. The default value is 3 and relates voxel-size to a mm-coordinate system used later on when displaying images in SPM. This should be changed to 1. This will create an image file with the file name appended by \texttt{F<frequency range>}. The image will have 3 dimensions (space(X,Y) and time). As with the TF data these images should be smoothed to accomodate spatial/temporal variability between subjects.
\\
\\
If when calling \texttt{Convert to images} you choose a non time-frequency data file, you will then be prompted to enter the output dimensions of the (interpolated scalp)  image that will be produced. Typically we suggest using 64. You will then be asked whether you want to interpolate or remove bad channels from your images. If you choose interpolate then the image will interpolate missing channels. This is the preferred option.  This will then create an image file in 3D where the dimensions are space (X,Y) and time. As before a separate image is created for each trial type. These images must be smoothed prior to second level analysis using the \textsc{smooth} function in the drop down ``Other'' menu. Once the images have been smoothed one can proceed to the second level analysis.
