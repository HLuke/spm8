\chapter{Analysis in sensor space \label{Chap:eeg:sensoranalysis}}

Second-level analyses for M/EEG are used to make inferences about the population from which subjects are drawn. This chapter describes how to perform second-level analyses of EEG/MEG or time-frequency (TF) data. This simply requires transforming data from \texttt{filename.mat} and \texttt{filename.dat}
format to image files (NIfTI format). Once the data are in image format second-level analyses for M/EEG are procedurally identical to 2nd level analyses for fMRI.
We therefore refer the reader to the fMRI section for further details of this last step. Also the ``Multimodal face-evoked responses'' tutorial chapter ~\ref{Chap:data:multimodal} contains detailed examples of sensor-level analysis. 
\\
In the drop down ``Other'' menu, select the function \texttt{Convert to images}. This will ask you to select the \texttt{filename.mat} of the data you would like to convert to images.
\\
\subsection{Data in time domain}
If when calling \texttt{Convert to images} you choose an epoched or averaged dataset, you will then be prompted to enter the output dimensions of the interpolated scalp image that will be produced. Typically, we suggest using 64. You will then be asked whether you want to interpolate or remove bad channels from your images. If you choose interpolate then the image will interpolate missing channels. This is the preferred option. If you choose to remove bad channels there will be 'holes' in the resulting images. A directory will be created with the same name as the input dataset. In this directory there will be a subdirectory for each trial type. These directories will contain 3D image files where the dimensions are space (X,Y) and time. In the case of averaged dataset a single image is put in each directory. In the case of epoched dataset there will be an image for each trial. 
\\
\subsubsection{Averaging over time}
If you know in advance the time window you are interested in (for instance if you are interested in a well characterized ERP/ERF peak) you can average over this time window to create a 2D image with just the spatial dimensions. This is done by pressing the \texttt{Specify 1-st level} button (the name is historical as this operation is remotely related to 1-st level analysis of fMRI data). You will be asked to specify the time window (in ms) and select one or more space-time images and the output directory. 
\\
\subsubsection{Masking}
When you set up your statistical analysis, it might be useful to use an explicit mask for two reasons. Firstly, if you smooth the images, the areas outside the scalp will also have some values and they will be included in the analysis by default if you don't use a mask. Secondly, you might want to limit your time window. This can also be done by using small volume correction afterwards, but using a mask might be more convenient if you have a fixed time window of interest. Such a mask can be created by selecting \texttt{Mask images} from ``Other'' dropdown menu. You will be asked to provide one unsmoothed image to be used as atemplate for the mask. This can be any of the images you exported. Then you will be asked to specify the time window of interest and the name for the output mask file. 
\\
\subsection{Time-frequency data}
If you supply a time-frequency dataset as the input, you will first be asked to reduce your data from a 4D data (space(X,Y), time, frequency) to either a 3D image (space(X,Y), time) or a 2D time-frequency image (time, frequency). The prompt asks you over which dimension you wish to average your data.
\\
If you select ``channels'' you will be asked to select the channels you wish to average over. Next you will be prompted for a ``region number''. This is just a tag that allows you to average over different channels to create different analyses for more than one region. The image will be created in a new directory with the name \texttt{<region number>ROI\_TF\_trialtype\_<condition number>}. The image that will be created will be a 2D image where the dimensions are time and frequency. 
\\
If you select ``frequency'' you will be asked to specify the frequency range over which you wish to average. The power is then averaged over the specified frequency band to produce channel waveforms. These waveforms are saved in a new time domain M/EEG dataset whose name starts with \texttt{F<frequency range>}. This dataset can be reviewed and further processed in the same way as ordinary datasets with waveforms (of course source reconstruction or DCM are will not work for it). Once this dataset is generated it is automatically exported to images the same way as data in time domain (see above). 
\\
\subsection{Smoothing}
The images generated from M/EEG data must be smoothed prior to second level analysis using the \textsc{Smooth images} function in the drop down ``Other'' menu. Smoothing is necessary to accommodate spatial/temporal variability between subjects and make the images better conform to the assumptions of random field theory. The dimensions of the smoothing kernel are specified in the units of the original data ([mm mm ms] for space-time, [Hz ms] for time-frequency). The general guiding principle for deciding how much to smooth is the matched filter idea, which says that the smoothing kernel should match the data feature one wants to enhance. Therefore, the spatial extent of the smoothing kernel should be more or less similar to the extent of the dipolar patterns that you are looking for (probably something of the order of magnitude of several cm).  In practice you can try to smooth the images with different kernels designed according to the principle above and see what looks best. Smoothing in time dimension is not always necessary as filtering the data has the same effect.
\\

Once the images have been smoothed one can proceed to the second level analysis.